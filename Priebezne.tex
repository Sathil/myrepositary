\documentclass[]{article}
\usepackage[utf8]{inputenc}
\usepackage{geometry}
\usepackage{amsmath}
\usepackage{amsfonts}
\usepackage{amssymb}
\usepackage{fancyhdr}


\pagenumbering{arabic}
\setcounter{page}{520}
\pagestyle{fancy}
\fancyhf{}
\fancyhead[R]{\thepage}


\begin{document}
\title {\textbf{Leveraging Microblogs for Resource Ranking}}
\author{Tomáš Majer and Marián Šimko } 
\date {\small  {{\fontfamily{pcr}\selectfont Faculty of Informatics and Information Technologies, Slovak University of
Technology, Ilkovičova 3, 842 16, Bratislava 4}}}
\maketitle 



\begin{center} \parbox{350pt}{\textbf {Abstract.} In order to compute page rankings, search algorithms primarily
utilize information related to page content and link structure.
Microblog as a phenomenon of today provides additional, potentially
relevant, information – short messages often containing hypertext links
to web resources. Such source is particularly valuable when considering
a temporal aspect of information, which is being published every second.
In this paper we present a method for resource ranking based on Twitter
data structure processing. We apply various graph algorithms leveraging
the notion of a node centrality in order to deduce microblog-based resource
ranking. Our method ranks a microblog user based on his followers
count with respect to a number of (re)posts and reflects it into resource
ranking. The evaluation of the method showed that micro-based resource
ranking a) can not be substituted by a common form of an explicit user
rating, and b) has the great potential for search improvement.

\vspace{5mm} \textbf {Keywords:} microblog, Twitter, resource ranking, web search.
}
\end{center}
\section{Introduction}

Nowadays, people benefit from the Internet and its most important service –
the Web, which no longer consists only of static pages that people browse and
search information in. User generated content has become more important than
ever, what caused that a new medium for publishing short posts has emerged –
microblog.

Microblog is a lightweight form of traditional blog, where users publish only
brief reports. This phenomenon has become popular mainly thanks to microblog
Twitter. Therefore, the most common name for a user post is “tweet”. Microblog
introduced a completely new form of communication. Users became able to share
their actual feelings, experiences or opinions. Their posts are often related (and
explicitly linked) to entities, which typically represent personalities, products or
events. By aggregating different posts on various topics coming from a plethora
of users and various platforms, microblog became a valuable source of data.

It is possible to monitor current users’ opinions all around the world on microblog.
Different world events can be tracked [4] such as elections, an introduction
of a new product to the market, or revolutions. Typically it is a very current
and extensive user content. Such content can be processed and utilized in order
to improve access to information, e.g. by improving search. Currently the searchtakes into consideration criteria, where the most important are the content and
links structure. There are also other areas such as product review or rating,
where ranking can be utilized for filtering and sorting. However, those can be
subsequently (and unfairly) amended by owners of e-shops. Thinking of Twitter
as of a separate service, it contains un-moderated user data, which typically are
not as biased and are more objective.

In this paper we propose a method for leveraging microblogs for ranking
resources (typically web pages). The method is based on Twitter graph analysis
that results into resource ranking computation. The method is domain
independent as it processes a graph representation abstracting of the content
itself.

The rest of the paper is structured as follows. Section 2 summarizes work
related to microblogs and resource ranking. In section 3 we introduce a novel
ranking measure called TweetRank. The evaluation and a dataset acquired for
experiments are described in section 4. In section 5 we conclude our work.

\section{Related work}
There are many areas where microblog mining is beneficial. By textual processing
of user posts we can gather specific keywords for a particular user [22], or use
text classification methods to classify posts and users to specific topics [15]. It is
possible to use methods and techniques of opinion mining to find user opinions
[13] such as polarity of tweet.

Much work has been done in the area of microblog search and microblog
posts ranking [12,20]. Also the well known search engines like Google include
tweets in their search results, based on full text search. Twitter itself contains
a custom search engine allowing to search effectively through full text or tags.
Twitter can be used to find relevant recent resources over the Web and to identify
new and current trends in the world [5]. However, little is known about
how Twitter posts influence the relevancy of resources contained in the posts
and how this information can be utilized outside microblog. To our best knowledge,
we are not aware of any method directly using microblog posts for deriving
resource ranking and improving search in resources contained within those
posts.

Viewing Twitter as a huge network, it is natural to apply graph algorithms to
analyze its structure. General graph ranking algorithms such as PageRank, HITS
or NodeRanking can be applied to a graph representing microblog structure.
PageRank identifies nodes that have the greatest relevance in the graph. It can
be used to rank users by analyzing relationships [3]. NodeRanking also builds
on the random surfer model and, in addition to PageRank, considers weights of
links and the damping factor is not a constant but a node variable [14].

Some ranking algorithms have been already adopted for microblogs. Twitter-
Rank is based on PageRank [21]. Besides link structure, it considers topical similarity
between users.TunkRank is basic ranking method for social networks, which
are based on microblog principles [7]. It is directly correlated with an expected number of readers of a twitterer and inversely correlated with the number of users
he follows. TunkRank algorithm can be used to identify influential users.

\section{Microblog-Based Resource Ranking}
Twitter is a service combining elements of both social networking and microblogging
with certain specifics [8].While user profiles in social networks are connected
bi-directionally, connections on microblog have only one-way orientation. User
may follow other users in order to track their posts, but it does not necessarily
mean that also reverse links exist, i.e., a followed user may choose not to follow
ones that follow him. Twitter data analysis revealed that 80.5  of users
is followed by 80  of their followers [21]. It was also showed that posts that
users publish mostly depend on the number of users who follow them [8]. In
contrast with traditional blogs, post comments cannot be assigned directly to a
post. However, a new post linking to existing posts can be created, referred to
as retweet. Users have the option to setup visibility for each post and restrict
the “audience” of his tweets.

Twitter is mostly used by users who actively use the web. It is therefore
natural that a 22  of their tweets contain links to websites [2]. Due to the
limited length of the contributions many of the services that shorten the web
address are used, called URL-shortening services.

The aforementioned characteristics can be used to construct a graph representation
(see Fig. 1). Twitter graph contains three types of entities: users, tweets
and resources. There are some tweets on Twitter, which are not connected to resources
(typically web pages). The edges between users represent a unidirectional
followership relation. Users are also connected with tweets they post. Tweets can
be linked to users, another tweets and pages. Relations between tweets represent
a re-tweet of a “parent” tweet.

\subsection {TweetRank Computation}
We believe the structure of microblog posts (represented by the graph) can be
leveraged to determine ranking of resources referenced in microblog tweets.

We propose a method for computing a resource rank, which we call Tweet-
Rank. Basically, TweetRank is derived from tweet topology. The method is based
on the existing graph analysis algorithms, while extending them with microblog
specific features. The principle of the method lies in user ranking estimation and
propagating such rank via graph relationships into web resources.

In order to obtain resource ranking, we first compute user ranking based on
modified TunkRank algorithm [6]. In contrast with TunkRank, we do not use
static constant transition probability, but calculate dynamic coefficient for each
user based on a number of his followers and tweets. For a user rank calculation
we proposed the following equation:

where UserRank(u) represents rank of a user u, followers(u) represents set of
followers of user u and tweets(u) is set of tweets of user u.

After computing a rank for each user, we use the UserRank to rank user
tweets. Users with higher UserRank post tweets that are more important and, as
a result, resources they share in those tweets are more relevant. The computation
is iterative: UserRank is based on UserRank of all user followers. The more
high-ranked followers a user has, the higher UserRank of this user is. In the
beginning we assign same UserRank value to every user. In order to obtain
final UserRank, we perform more UserRank computations until changes in
user ranks do not exceed a specified (very low) threshold.

For calculating a relevance of a tweet, we use the following computation:
where TR(t) = TweetRelevance(t) is relevance of tweet t, Author(t) is author
of tweet t. TweetRelevance represents relative rank of every tweet. It is based
on UserRank which is adjusted according to the number of tweets of the author.

After obtaining a rank for each tweet, we compute ranks of tweets, which
link to resources, and derive resource rankings. When a tweet is re-tweeted, we
increase a rank for a resource, as we explained earlier. Finally, TweetRank for
a resource r is defined as an aggregation of ranks of all tweets that point to the
resource r:
where TweetRank(r) is TweetRank of a resource r, tweetsURL is set of tweets
containing URL and retweets(t) is set of retweets of tweet t.

For example, consider graph on Fig 1. Assuming user rankings UserRank(U1)
=6, UserRank(U2)=2, UserRank(U3) = 10 we calculate TweetRank of resource
R4 as follows: TweetRank(R4) = TR(T 5)+TR(T 5)TR(T6)+TR(T 6) = 2/3+
2/3  10/2 + 10/2 = 8.9 (normalisation omitted). Similarly, TweetRank(R2) =
5.6. The higher user ranking reflects into the higher ranking of a resource.
The proposedmethod is a novel graph analysis method for computing rankings
for microblog-based networks such as Twitter. For computing resource ranks it
considers microblog specific concepts: unidirectional relation of followership and
ability to re-tweet already tweeted post.

\section {Evaluation}
In order to evaluate the proposed method, we conducted two experiments. We
collected Twitter dataset with tweets containing links and users relations. In
the first experiment we rank YouTube videos and compare results based on
TweetRank and user rank from YouTube.

In the second experiment we evaluate our ranking method by incorporating
it in searching. We present an approach where resources are sorted based on
combination of fulltext search with microblog-based rating.

\subsection {Dataset}
We collected and joined two datasets to be used in experiments. First, we prepared
a dataset of Twitter data by using 140kit service1. The dataset contains
tweets and user profiles from 18th to 24th September 2009. There are 1,997,446
tweets and 367,824 user profiles in the dataset. User profiles contain basic user
data such as the count of his followers or the count of his total tweets. Dataset
contains tweets in 6 different languages. 85 % of tweets are in English. To calculate
ratings of users it is necessary to create connections between the users
(for the calculation of user rank we need to know user rank of all of his followers).
However, there were no such relations between users in obtained dataset.
In order to calculate the ratings of users, Twitter API can be used directly for
downloading a list of followers for a particular user. Unfortunately, this process
is slow and Twitter limits the number of requests (350 per hour). Therefore,
we used additional dataset, which contains information about user relations
[10]. The dataset contains only links between users based on followers relationship
and was created for three months from early June 2009 until the end of
September 2009. This dataset contains users, who are part of the first dataset.
The dataset contains 1,468,365,182 connections between 40,103,281 users. The
dataset together with a full description can be downloaded from the website of
the work2.

There are 1,150,168 unique web links in the dataset. By using our method for
resource ranking, we computed TweetRank for each resource determined by a link in the dataset. The distribution of the number of resources having particular
TweetRank value is depicted in Fig. 2. It follows a power-law distribution.

Further analysis of the dataset revealed that as much as 3  of links points
to YouTube videos. YouTube videos also contain explicit rating given by users.
In the first experiment we assess to what extent TweetRank relates with rating
coming from the YouTube site.

\subsection {Comparison with YouTube User Rank}

In addition to watching a video on YouTube, users can also comment the video,
they respond to comments of others and they vote. A vote can be positive or
negative. For the purpose of the experiment we downloaded the user votes of 605
videos.We transformed each vote into a rating calculated as a difference between
positive and negative votes and normalized it according to TweetRank rankings.
After that, we compared user rating to TweetRank ranking. We assumed that
the ranking of videos using TweetRank will be similar to that of the votes on
YouTube, as both ratings come from the crowd. However, we did not find a
correlation between both rankings (see Fig. 3).

The performed experiment showed that the evaluation of video-based ratings
of users on YouTube is not correlated with the TweetRank (correlation coefficient
r = 0.02). Although both approaches are basically based on rating of users,
results differ. We see the main difference in results in unequal length of ranking
time period. Videos on YouTube are available from the moment of their publication
until the author decides to remove the video (which occurs very rarely).
Long period YouTube rating is based on continuous rating through years, while
TweetRank ranks are computed based on user posts collected during a period
in September 2009 as described earlier. As we were not able to asses TweetRank
rankings, we decided to compare the rankings with user rankings in the period
closer to that from dataset in terms of both time distance and time length.

We created an application for collecting user ratings by enabling them to see
video and enabling them to enter a rank. The application offered to visitors 5 YouTube videos coming from the acquired dataset. The visitors were instructed
to rate the videos on a scale from 1 to 5, where 1 stood for best and 5 for
the worst rate. Users were given simple guidelines for evaluating videos: they
should rate videos according to how they like them. There were no other criteria
suggested they should consider. 70 different users participated in the experiment.
They were non-technically oriented, ordinary web users and thus, in the
most cases, also users of YouTube. We collected 680 responses together. The
participants were approached through social networks Facebook and microblog
Twitter.

Based on user voting, we created two comparisons. We sorted videos based
on average video rating computed according the following formula:
where AV R(v) represents video rating computed as an average rating from a set
of all video ratings H(v) of video v.
The results of voting showed that videos got rather lower user rank values in
general (see Table 1). When compared to the TweetRank results (Fig. 2), we
see similar characteristics: in both cases many videos got low rates and only a
few videos were assigned high rates. The user voting results were compared with
TweetRank rankings. We computed the Kendall rank correlation coefficient τ in
order to measure the association between both results.We obtained τ = -0.12519
(the resulting value is negative because user rank evaluation method has been
reversed). We also calculated the correlation coefficient r = 0.387.

As each user had to rate five videos at once in the created application, he
created own subjective ordering of the videos. Average difference between two
adjacent videos in ordering by TweetRank is 0.0025. We also considered a deviation
of rating in order to observe the change of two orderings (5-tuplets)
match ratio. One ordering was based on user ranks and the other was based on
Tweet-Rank (of the same videos). Deviation we define as a number, which we
need to add to any TweetRank of video from 5-tuplet of videos in order to make
changed TweetRank ordering match the 5-tuplet of videos composed by a user
(Fig. 4).

We see that finding a match in 5-tuple of user rating and TweetRank is
difficult, because disagreement even in one place makes a comparison of orderings
unsuccessful. Therefore, we also made another comparison, where we compared
pairs of videos rather than 5-tuples. Similarly, we considered deviation in order
to observe match ratio change (Fig. 5).

The number of matched pairs is far higher than in the case when comparing
5-tuples. Results show that there is more then 60 % of matches without considering
deviation. Although a match ratio increases with a growing deviation, there is
still much difference in agreement on orderings. We believe the explicit user rating
lacks information (let us name it a context) that was induced from microblog.
Microblog topology and relationships between all entities represents potentially  valuable information with regards to resources contained in blog posts. In order
to further analyze obtained TweetRank rankings, we conducted an experiment
related to search.

\subsection {Sorting Search Results with TweetRank}
In the experiment we used TweetRank for sorting search results. For the experiment
we selected a part of the acquired dataset, which contained the randomly
chosen 20,000 unique links to web resources. We indexed the resources using the
SOLR search platform, which utilizes the core Apache Lucene search3. We extended
document representation in order to incorporate calculated TweetRank,
allowing us to sort search results by that rank.

We proposed a hybrid approach for scoring computation combining internal
ranking of SOLR search engine and TweetRank. As a result, search yields
relevant documents (containing keywords from a query), which are sorted by
TweetRank. Top-k documents are selected for presentation on a result page.

We performed search with 20 randomly selected queries from well understood
domain and manually evaluated the results. In order to demonstrate the results
we have selected the keyword “apple”. Top-5 documents ranked by both Lucene
original ranking and our approach utilizing TweetRank are shown in Table 2.We
see fromthe example that the hybrid approach yields ordering, which prefers pages
containing information about releasing new products. This kind of pages also
ranked better with other queries we tried. Such behavior of rank originates in the
nature of microblogs, where people often react to world or local news [5]. However,
an important observation is that TweetRank-based ordering does not match
a chronological rank of resources, i.e., publication time of information on a page.
This leads us to conclusion that microblog-based ranking is especially useful when
ranking a set of resource created within a specific time window, as it can not be replaced
by temporal metatada of resources itself (creation time, change time, etc.).

\section {Conclusions}
The amount and nature of non-moderated user generated data emerging from
microblog made it a potentially relevant and powerful data source that can be
utilized for a variety of tasks. We believe that microblog data mining can improve
approaches to domain modeling [16,17] and user modeling [1,9] potentially
resulting in improved web search [18] or personalization [19,11]. It is important
to note that microblog has certain restrictions, typically related to a limited
size of a post or too specific language, which make microblog content processing
more difficult. However, its popularity and spread result into increased number
of posts every day. Nearly one quarter of posts contains URL to a web page [2].
Increasing popularity of microblog also increases a number of microblog posts,
which can be utilized for web search improvement. This makes microblog data
extremely perspective. Not only for search improvement at the Web scale, but
also in domains, where microblog emerges as a tool for instant discussions among
communities of the specialized interest.

In this paper we proposed a novel method for ranking resources referenced
by microblog users in their posts. We particularly focused on microblog Twitter
as it is both public and widely adopted. The principle of our approach lies in
ranking microblog users and reflecting this rank in resources present in posts they
publish. We utilize user ranking method TunkRank and extend it by considering
additional microblog specific features like re-tweet, which improves user ranking
computation in relation to posts a user publishes.

In order to evaluate the proposed method for resources ranking we conducted
two experiments. The first experiment we performed in the domain of user videos,
where we compared ratings obtained from YouTube website with ranking obtained
by our method. We found no correlations in both measures, what can be
explained by temporal characteristics of underlying data: while microblog-based
ratings were created based on one month sample of tweets, original user voting
at YouTube site is present for years. Therefore, we created own application
for collecting a different form of explicit user rating: relative ordering. The final
user ratings were compared with the rank obtained from microblog Twitter using
our method.We were able to see similarities between the measures, but there still was not sufficient number of agreements on rankings. We interpret this as a
presence of additional potentially valuable information originating in microblog
topology that can not be reflected in explicit user ratings.

In the second experiment we used TweetRank for sorting search results. We
combined document score computation by considering both a traditional statistical
measure and TweetRank obtained by the method we proposed.By performing
a set of search scenarios we assessed how the search results were reordered.
Results showed that combined scoring computation prefers pages containing
relatively new information. Such behavior of rank originates in the nature of
microblogs, where people often react instantly. On the other hand, TweetRankbased
ordering does not match a chronological rank of resources, i.e., resources
in top-k search results are not new in terms of time only, but also sorted with
regards of user attitudes expressed through microblog posts they write.

In conclusion, we consider results of our research very promising and perspective.
As a part of a future work we plan to conduct an experiment in a real world
setting by incorporating users, who will assess obtained search results. Links in
microblog posts are typically accompanied with user comments, which can be
considered as resource annotations. Our long term goal is to extend our method
with microblog posts content analysis in order to further improve obtained resource
rankings and possibly combine it with semantic search approaches [18].

\textbf{Acknowlegdments.} This work was partially supported by the grants VG1/
0675/ 11/2011-2014, KEGA 028-025STU-4/2010, APVV-0208-10 and it is the
partial result of the Research \& Development Operational Programme for the
project Research of methods for acquisition, analysis and personalized conveying
of information and knowledge, ITMS 26240220039, co-funded by the ERDF.

\begin{thebibliography}{9}

\bibitem{lamport94}
  Barla, M., Bielikova, M.: OrdinaryWeb Pages as a Source for Metadata Acquisition
for Open Corpus User Modeling. In: Proc. of WWW/Internet, pp. 227–233. IADIS
Press (2010)
\bibitem{b2}
Boyd, D.M., Golder, S., Lotan, G.: Tweet, Tweet, Retweet: Conversational Aspects
of Retweeting on Twitter. In: 43rd Hawaii International Conference on System
Sciences, pp. 1–10. IEEE (2010)
\bibitem{b3}
Brin, S., Page, L.: The anatomy of a large-scale hypertextual web search engine.
In: Proc. of the 7th Int. World Wide Web Conf. (1998)
\bibitem{b4}
Diakopoulos, N.A., Shamma, D.A.: Characterizing Debate Performance via
Aggregated Twitter Sentiment. In: Proc. of the 28th Int. Conf. on Human Factors
in Computing Systems, pp. 1195–1198. ACM (2010)
\bibitem{b5}
Dong, A.: Time is of the essence: improving recency ranking using Twitter data.
In: Proc. of the 19th Int. Conf. on World Wide Web, pp. 331–340. ACM (2010)
\bibitem{b6}
Gayo-Avello, D., Brenes, D.J.: Overcoming Spammers in Twitter: A Tale of Five
Algorithms. ir.ii.uam.es, pp. 41–52 (2010)
\bibitem{b7}
Gayo-Avello, D.: Nepotistic Relationships in Twitter and their Impact on Rank
Prestige Algorithms. In: Arxiv preprint, arXiv:1004.0816 (2010)
\bibitem{b8}
Huberman, B.A., Romero, D.M.: Social networks that matter: Twitter under the
microscope. In: Arxiv preprint, arXiv:0812.1045v1 (2009)
\bibitem{b9}
Kramár, T., Barla, M., Bieliková, M.: PeWeProxy: A Platform for Ubiquitous
Personalization of the ”Wild” Web. In: UMAP 2011: Adjunct Proc. of the 19th
Int. Conf. on User Modeling, Adaptation and Personalization. Demo., pp. 7–9
(2011)
\bibitem{b10}
Kwak, H., Lee, C., Park, H.: What is Twitter, a Social Network or a News Media?
In: Proceedings of the 19th International Conference on World Wide Web, Raleigh,
pp. 591–600. ACM (2010)
\bibitem{b11}
Labaj, M.: Information Sciences and Technologies Bulletin of the ACM Slovakia.
Special Section on Student Research in Informatics and Information Technologies
3(2), 76–78 (2011)
\bibitem{b12}
Nagmoti, R., Teredesai, A., De Cock, M.: Ranking Approaches for Microblog
Search. In: Proc. of the 2010 IEEE/WIC/ACM Int. Conf. on Web Intelligence
and Intelligent Agent Technology, vol. 01, pp. 153–157. IEEE Computer Society,
Washington, DC (2010)
\bibitem{b13}
Pandey, V., Iyer, C.: Sentiment analysis of microblogs (2009),
http://www.stanford.edu/class/cs229/proj2009/PandeyIyer.pdf (accessed
October 05, 2011)
\bibitem{b14}
Pujol, J.M., Sangesa, R., Delgado, J.: Extracting Reputation in Multi Agent Systems
by Means of Social Network Topology. In: Proc. of the First Int.l Joint Conf.
Autonomous Agents and Multiagent Systems, pp. 467–474 (2002)
\bibitem{b15}
Ramage, D., Dumais, S., Liebling, D.: Characterizing Microblogs with Topic Models.
In: Proc. of Int. AAAI Conf. on Weblogs and Social Media, pp. 130–137. AAAI
Press (2010)
\bibitem{b16}
Simko, J., Tvarožek, M., Bieliková, M.: Little Search Game: Term Network Acquisition
via a Human Computation Game. In: HT 2011: Proc. of the 22nd ACM
Conf. on Hypertext and Hypermedia, pp. 57–61. ACM, New York (2011)
\bibitem{b17}
Šimko, J.: Augmenting Human Computed Lightweight Semantics. Information Sciences
and Technologies Bulletin of the ACM Slovakia, Special Section on Student
Research in Informatics and Information Technologies 3(2), 116–118 (2011)
\bibitem{b18}
Šimko, M., Bieliková, M.: Improving Search Results with Lightweight Semantic
Search. In: Grobelnik, M., Mika, P., Douc, T.T., Wang, H. (eds.) Proc. of the
Workshop on Semantic Search, SemSearch 2009 at the 18th Int. World Wide Web
Conference, WWW 2009, Madrid, Spain. CEUR, vol. 491, pp. 53–54 (2009)
\bibitem{b19}
Šimko, M., Barla, M., Bieliková, M.: ALEF: A Framework for AdaptiveWeb-Based
Learning 2.0. In: Reynolds, N., Turcsányi-Szabó, M. (eds.) KCKS 2010. IFIP AICT,
vol. 324, pp. 367–378. Springer, Heidelberg (2010)
\bibitem{b20}
Teevan, J., Ramage, D., Morris, M.R.: TwitterSearch: a comparison of microblog
search and web search. In: Proc. of the Fourth ACM Int. Conf. on Web Search and
Data Mining, WSDM 2011, pp. 35–44. ACM, New York (2011)
\bibitem{b21}
Weng, J., Lim, E.P., Jiang, J., He, Q.: TwitterRank: Finding Topic-sensitive Influential
Twitterers. In: Proc. of the Third ACM Int. Conf. on Web Search and Data
Mining, pp. 261–270. ACM (2010)
\bibitem{b22}
Wu, W., Zhang, B., Ostendorf, M.: Automatic generation of personalized annotation
tags for Twitter users. In: Proc. HLT 2010 Human Language Technologies:
The 2010 Annual Conf. of the North American Chapter of the Association for
Computational Linguistics, pp. 689–692. ACM (2010)

\end{thebibliography}
\end{document}